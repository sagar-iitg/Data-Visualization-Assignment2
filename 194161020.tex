% Options for packages loaded elsewhere
\PassOptionsToPackage{unicode}{hyperref}
\PassOptionsToPackage{hyphens}{url}
%
\documentclass[
]{article}
\usepackage{lmodern}
\usepackage{amssymb,amsmath}
\usepackage{ifxetex,ifluatex}
\ifnum 0\ifxetex 1\fi\ifluatex 1\fi=0 % if pdftex
  \usepackage[T1]{fontenc}
  \usepackage[utf8]{inputenc}
  \usepackage{textcomp} % provide euro and other symbols
\else % if luatex or xetex
  \usepackage{unicode-math}
  \defaultfontfeatures{Scale=MatchLowercase}
  \defaultfontfeatures[\rmfamily]{Ligatures=TeX,Scale=1}
\fi
% Use upquote if available, for straight quotes in verbatim environments
\IfFileExists{upquote.sty}{\usepackage{upquote}}{}
\IfFileExists{microtype.sty}{% use microtype if available
  \usepackage[]{microtype}
  \UseMicrotypeSet[protrusion]{basicmath} % disable protrusion for tt fonts
}{}
\makeatletter
\@ifundefined{KOMAClassName}{% if non-KOMA class
  \IfFileExists{parskip.sty}{%
    \usepackage{parskip}
  }{% else
    \setlength{\parindent}{0pt}
    \setlength{\parskip}{6pt plus 2pt minus 1pt}}
}{% if KOMA class
  \KOMAoptions{parskip=half}}
\makeatother
\usepackage{xcolor}
\IfFileExists{xurl.sty}{\usepackage{xurl}}{} % add URL line breaks if available
\IfFileExists{bookmark.sty}{\usepackage{bookmark}}{\usepackage{hyperref}}
\hypersetup{
  hidelinks,
  pdfcreator={LaTeX via pandoc}}
\urlstyle{same} % disable monospaced font for URLs
\usepackage[margin=1in]{geometry}
\usepackage{color}
\usepackage{fancyvrb}
\newcommand{\VerbBar}{|}
\newcommand{\VERB}{\Verb[commandchars=\\\{\}]}
\DefineVerbatimEnvironment{Highlighting}{Verbatim}{commandchars=\\\{\}}
% Add ',fontsize=\small' for more characters per line
\usepackage{framed}
\definecolor{shadecolor}{RGB}{248,248,248}
\newenvironment{Shaded}{\begin{snugshade}}{\end{snugshade}}
\newcommand{\AlertTok}[1]{\textcolor[rgb]{0.94,0.16,0.16}{#1}}
\newcommand{\AnnotationTok}[1]{\textcolor[rgb]{0.56,0.35,0.01}{\textbf{\textit{#1}}}}
\newcommand{\AttributeTok}[1]{\textcolor[rgb]{0.77,0.63,0.00}{#1}}
\newcommand{\BaseNTok}[1]{\textcolor[rgb]{0.00,0.00,0.81}{#1}}
\newcommand{\BuiltInTok}[1]{#1}
\newcommand{\CharTok}[1]{\textcolor[rgb]{0.31,0.60,0.02}{#1}}
\newcommand{\CommentTok}[1]{\textcolor[rgb]{0.56,0.35,0.01}{\textit{#1}}}
\newcommand{\CommentVarTok}[1]{\textcolor[rgb]{0.56,0.35,0.01}{\textbf{\textit{#1}}}}
\newcommand{\ConstantTok}[1]{\textcolor[rgb]{0.00,0.00,0.00}{#1}}
\newcommand{\ControlFlowTok}[1]{\textcolor[rgb]{0.13,0.29,0.53}{\textbf{#1}}}
\newcommand{\DataTypeTok}[1]{\textcolor[rgb]{0.13,0.29,0.53}{#1}}
\newcommand{\DecValTok}[1]{\textcolor[rgb]{0.00,0.00,0.81}{#1}}
\newcommand{\DocumentationTok}[1]{\textcolor[rgb]{0.56,0.35,0.01}{\textbf{\textit{#1}}}}
\newcommand{\ErrorTok}[1]{\textcolor[rgb]{0.64,0.00,0.00}{\textbf{#1}}}
\newcommand{\ExtensionTok}[1]{#1}
\newcommand{\FloatTok}[1]{\textcolor[rgb]{0.00,0.00,0.81}{#1}}
\newcommand{\FunctionTok}[1]{\textcolor[rgb]{0.00,0.00,0.00}{#1}}
\newcommand{\ImportTok}[1]{#1}
\newcommand{\InformationTok}[1]{\textcolor[rgb]{0.56,0.35,0.01}{\textbf{\textit{#1}}}}
\newcommand{\KeywordTok}[1]{\textcolor[rgb]{0.13,0.29,0.53}{\textbf{#1}}}
\newcommand{\NormalTok}[1]{#1}
\newcommand{\OperatorTok}[1]{\textcolor[rgb]{0.81,0.36,0.00}{\textbf{#1}}}
\newcommand{\OtherTok}[1]{\textcolor[rgb]{0.56,0.35,0.01}{#1}}
\newcommand{\PreprocessorTok}[1]{\textcolor[rgb]{0.56,0.35,0.01}{\textit{#1}}}
\newcommand{\RegionMarkerTok}[1]{#1}
\newcommand{\SpecialCharTok}[1]{\textcolor[rgb]{0.00,0.00,0.00}{#1}}
\newcommand{\SpecialStringTok}[1]{\textcolor[rgb]{0.31,0.60,0.02}{#1}}
\newcommand{\StringTok}[1]{\textcolor[rgb]{0.31,0.60,0.02}{#1}}
\newcommand{\VariableTok}[1]{\textcolor[rgb]{0.00,0.00,0.00}{#1}}
\newcommand{\VerbatimStringTok}[1]{\textcolor[rgb]{0.31,0.60,0.02}{#1}}
\newcommand{\WarningTok}[1]{\textcolor[rgb]{0.56,0.35,0.01}{\textbf{\textit{#1}}}}
\usepackage{graphicx,grffile}
\makeatletter
\def\maxwidth{\ifdim\Gin@nat@width>\linewidth\linewidth\else\Gin@nat@width\fi}
\def\maxheight{\ifdim\Gin@nat@height>\textheight\textheight\else\Gin@nat@height\fi}
\makeatother
% Scale images if necessary, so that they will not overflow the page
% margins by default, and it is still possible to overwrite the defaults
% using explicit options in \includegraphics[width, height, ...]{}
\setkeys{Gin}{width=\maxwidth,height=\maxheight,keepaspectratio}
% Set default figure placement to htbp
\makeatletter
\def\fps@figure{htbp}
\makeatother
\setlength{\emergencystretch}{3em} % prevent overfull lines
\providecommand{\tightlist}{%
  \setlength{\itemsep}{0pt}\setlength{\parskip}{0pt}}
\setcounter{secnumdepth}{-\maxdimen} % remove section numbering

\title{\textcolor{black}{Assignment 2: Data Visualization}}
\author{\textcolor{black}{Rohit Jain 	 | 	194161020}}
\date{}

\begin{document}
\maketitle

\begin{center}\rule{0.5\linewidth}{0.5pt}\end{center}

\hypertarget{question-1}{%
\section{Question 1:}\label{question-1}}

\textbf{Load the data gapminder and analyze different columns of
data.Plot life expectancy over time for each country. Do you think the
plot is meaningful? Justify your answer in markdown}

\begin{Shaded}
\begin{Highlighting}[]
\NormalTok{p <-}\StringTok{ }\KeywordTok{ggplot}\NormalTok{(}\DataTypeTok{data =}\NormalTok{ gapminder,}
            \DataTypeTok{mapping =} \KeywordTok{aes}\NormalTok{(}\DataTypeTok{x =}\NormalTok{ year,}
                          \DataTypeTok{y =}\NormalTok{ lifeExp, }\DataTypeTok{col=}\NormalTok{country))}
\NormalTok{p }\OperatorTok{+}\StringTok{ }\KeywordTok{geom_line}\NormalTok{() }\OperatorTok{+}
\StringTok{  }\KeywordTok{guides}\NormalTok{(}\DataTypeTok{col=}\OtherTok{FALSE}\NormalTok{) }\OperatorTok{+}\StringTok{ }
\StringTok{  }\KeywordTok{theme_classic}\NormalTok{() }\OperatorTok{+}\StringTok{ }
\StringTok{  }\KeywordTok{labs}\NormalTok{(}\DataTypeTok{x=}\StringTok{"Year"}\NormalTok{, }\DataTypeTok{y=}\StringTok{"Life Expectancy"}\NormalTok{, }
       \DataTypeTok{title=}\StringTok{" Life Expectancy over years!"}\NormalTok{,}
       \DataTypeTok{subtitle=}\StringTok{"Country Wise"}\NormalTok{, }
       \DataTypeTok{caption=}\StringTok{"Dataset:gapminder"}\NormalTok{)}\OperatorTok{+}\StringTok{ }
\StringTok{  }\NormalTok{theme}
\end{Highlighting}
\end{Shaded}

\includegraphics{194161020_files/figure-latex/unnamed-chunk-1-1.pdf}

The plot is not very meaningful since we cannot differentiate the
particular pattern for any country. The plot looks rough and messy and
hence not very meaningful.

\hypertarget{question-2}{%
\section{Question 2}\label{question-2}}

\textbf{I. Can you improve the plot from the first question by faceting
the data? Which is the most appropriate variable (column) to facet the
data? Plot and justify.}

\begin{Shaded}
\begin{Highlighting}[]
\NormalTok{p <-}\StringTok{ }\KeywordTok{ggplot}\NormalTok{(}\DataTypeTok{data =}\NormalTok{ gapminder,}
            \DataTypeTok{mapping =} \KeywordTok{aes}\NormalTok{(}\DataTypeTok{x =}\NormalTok{ year,}
                          \DataTypeTok{y =}\NormalTok{ lifeExp, }\DataTypeTok{col=}\NormalTok{country))}
\NormalTok{p }\OperatorTok{+}\StringTok{ }\KeywordTok{geom_line}\NormalTok{() }\OperatorTok{+}\StringTok{ }
\StringTok{  }\KeywordTok{facet_wrap}\NormalTok{(}\OperatorTok{~}\NormalTok{continent)}\OperatorTok{+}
\StringTok{  }\KeywordTok{guides}\NormalTok{(}\DataTypeTok{col=}\OtherTok{FALSE}\NormalTok{)}\OperatorTok{+}
\StringTok{  }\KeywordTok{theme_bw}\NormalTok{()}\OperatorTok{+}
\StringTok{  }\NormalTok{theme}\OperatorTok{+}
\StringTok{  }\KeywordTok{labs}\NormalTok{(}\DataTypeTok{x=}\StringTok{"Year"}\NormalTok{, }\DataTypeTok{y=}\StringTok{"Life Expectancy"}\NormalTok{, }
       \DataTypeTok{caption=}\StringTok{"Dataset:gapminder"}\NormalTok{,}
       \DataTypeTok{subtitle=}\StringTok{"Continent Wise"}\NormalTok{,}
       \DataTypeTok{title=}\StringTok{" Life Expectancy over years for each country!"}\NormalTok{)}
\end{Highlighting}
\end{Shaded}

\includegraphics{194161020_files/figure-latex/unnamed-chunk-2-1.pdf}
\textbf{``Continent''} turns out to be the best column for faceting the
data into groups because it is categorical in nature while other
remaining columns have continuous values. Categorical variable makes it
easy for grouping the data together.

\textbf{II. We can facet the data based on more than one variable, Use
gss sm data to plot a smoothed scatter plot showing the relationship
between the age of the respondent and the number of children they have.
Facet this relation based on race and degree. Also, in markdown,
describe your observation from the plot in brief.}

\begin{Shaded}
\begin{Highlighting}[]
\NormalTok{p<-}\KeywordTok{ggplot}\NormalTok{(}\DataTypeTok{data=}\NormalTok{gss_sm, }\KeywordTok{aes}\NormalTok{(}\DataTypeTok{x=}\NormalTok{age, }\DataTypeTok{y=}\NormalTok{childs))}

\NormalTok{p}\OperatorTok{+}\StringTok{ }\KeywordTok{geom_point}\NormalTok{(}\KeywordTok{aes}\NormalTok{(}\DataTypeTok{col=}\StringTok{""}\NormalTok{))}\OperatorTok{+}
\StringTok{  }\KeywordTok{geom_smooth}\NormalTok{(}\DataTypeTok{se=}\OtherTok{FALSE}\NormalTok{, }\DataTypeTok{col=}\StringTok{"red"}\NormalTok{)}\OperatorTok{+}
\StringTok{  }\KeywordTok{facet_grid}\NormalTok{(race}\OperatorTok{~}\NormalTok{degree)}\OperatorTok{+}
\StringTok{   }\KeywordTok{guides}\NormalTok{(}\DataTypeTok{col=}\OtherTok{FALSE}\NormalTok{) }\OperatorTok{+}\StringTok{ }
\StringTok{  }\KeywordTok{theme_bw}\NormalTok{()}\OperatorTok{+}
\StringTok{  }\NormalTok{theme}\OperatorTok{+}
\StringTok{  }\KeywordTok{labs}\NormalTok{(}\DataTypeTok{x=}\StringTok{"Age"}\NormalTok{, }\DataTypeTok{y=}\StringTok{"Number of Children"}\NormalTok{, }
       \DataTypeTok{caption=}\StringTok{"Dataset:gss_sm"}\NormalTok{,}
       \DataTypeTok{subtitle=}\StringTok{"Based on race and degree"}\NormalTok{,}
       \DataTypeTok{title=}\StringTok{"Age of the respondents vs Number of children"}\NormalTok{)}
\end{Highlighting}
\end{Shaded}

\begin{verbatim}
## `geom_smooth()` using method = 'gam' and formula 'y ~ s(x, bs = "cs")'
\end{verbatim}

\includegraphics{194161020_files/figure-latex/unnamed-chunk-3-1.pdf}

In this graph we can see the relationship between the age of the
respondent and the number of children they have, grouped based on race
and degree. The first plot is basically about the respondents who belong
to white race and have left High School.

\hypertarget{question-3}{%
\section{Question 3}\label{question-3}}

\textbf{gss sm data contains the political view (polviews) variable.
Plot a bar graph, with the bar in the chart represented by different
political views.}

\begin{Shaded}
\begin{Highlighting}[]
\NormalTok{p<-}\KeywordTok{ggplot}\NormalTok{(}\DataTypeTok{data=}\NormalTok{gss_sm, }\KeywordTok{aes}\NormalTok{(}\DataTypeTok{x=}\NormalTok{polviews))}
\NormalTok{p}\OperatorTok{+}\KeywordTok{geom_bar}\NormalTok{(}\KeywordTok{aes}\NormalTok{(}\DataTypeTok{col=}\StringTok{""}\NormalTok{,}\DataTypeTok{fill=}\StringTok{""}\NormalTok{ ),}\DataTypeTok{alpha=}\FloatTok{0.6}\NormalTok{)}\OperatorTok{+}\StringTok{ }
\StringTok{  }\KeywordTok{guides}\NormalTok{(}\DataTypeTok{col=}\OtherTok{FALSE}\NormalTok{,}\DataTypeTok{fill=}\OtherTok{FALSE}\NormalTok{)}\OperatorTok{+}
\StringTok{  }\KeywordTok{theme_bw}\NormalTok{()}\OperatorTok{+}
\StringTok{  }\NormalTok{theme}\OperatorTok{+}
\StringTok{  }\KeywordTok{labs}\NormalTok{(}\DataTypeTok{x=}\StringTok{"Political Views"}\NormalTok{, }\DataTypeTok{y=}\StringTok{"Count"}\NormalTok{, }
       \DataTypeTok{title=}\StringTok{"Political views of the respondents!"}\NormalTok{, }
       \DataTypeTok{caption=}\StringTok{"Dataset:gss_sm"}\NormalTok{)}\OperatorTok{+}
\StringTok{  }\KeywordTok{theme}\NormalTok{(}\DataTypeTok{axis.text.x =} \KeywordTok{element_text}\NormalTok{(}\DataTypeTok{angle =} \DecValTok{45}\NormalTok{, }\DataTypeTok{hjust =} \DecValTok{1}\NormalTok{))}
\end{Highlighting}
\end{Shaded}

\includegraphics{194161020_files/figure-latex/unnamed-chunk-4-1.pdf}

\hypertarget{question-4}{%
\section{Question 4}\label{question-4}}

\textbf{Again using gss sm data, visualize the frequency plot, with the
bars representing different political views and each bar in the graph,
is further categorized by a different religion. Also, visualize
frequency plot faceted by variable bigregion}

\begin{Shaded}
\begin{Highlighting}[]
\NormalTok{p <-}\StringTok{ }\KeywordTok{ggplot}\NormalTok{(}\DataTypeTok{data =}\NormalTok{ gss_sm,}
            \DataTypeTok{mapping =} \KeywordTok{aes}\NormalTok{(}\DataTypeTok{x =}\NormalTok{ polviews, }
                          \DataTypeTok{fill =}\NormalTok{ religion))}
\NormalTok{p }\OperatorTok{+}\StringTok{ }\KeywordTok{geom_bar}\NormalTok{(}\KeywordTok{aes}\NormalTok{(}\DataTypeTok{y=}\NormalTok{..prop..))}\OperatorTok{+}
\StringTok{  }\KeywordTok{theme_classic}\NormalTok{()}\OperatorTok{+}
\StringTok{  }\NormalTok{theme}\OperatorTok{+}
\StringTok{  }\KeywordTok{labs}\NormalTok{(}\DataTypeTok{x=}\StringTok{"Political Views"}\NormalTok{, }\DataTypeTok{y=}\StringTok{"Frequency"}\NormalTok{, }
       \DataTypeTok{title=}\StringTok{"Political views of the respondents!"}\NormalTok{,}
       \DataTypeTok{subtitle=}\StringTok{"Religion wise"}\NormalTok{, }
       \DataTypeTok{caption=}\StringTok{"Dataset:gss_sm"}\NormalTok{)}\OperatorTok{+}
\StringTok{  }\KeywordTok{theme}\NormalTok{(}\DataTypeTok{axis.text.x =} \KeywordTok{element_text}\NormalTok{(}\DataTypeTok{angle =} \DecValTok{45}\NormalTok{, }\DataTypeTok{hjust =} \DecValTok{1}\NormalTok{))}
\end{Highlighting}
\end{Shaded}

\includegraphics{194161020_files/figure-latex/unnamed-chunk-5-1.pdf}

\begin{Shaded}
\begin{Highlighting}[]
\NormalTok{p <-}\StringTok{ }\KeywordTok{ggplot}\NormalTok{(}\DataTypeTok{data =}\NormalTok{ gss_sm,}
            \DataTypeTok{mapping =} \KeywordTok{aes}\NormalTok{(}\DataTypeTok{x =}\NormalTok{ polviews, }\DataTypeTok{fill =}\NormalTok{ religion))}
\NormalTok{p }\OperatorTok{+}\StringTok{ }\KeywordTok{geom_bar}\NormalTok{(}\KeywordTok{aes}\NormalTok{(}\DataTypeTok{y=}\NormalTok{..prop..))}\OperatorTok{+}
\StringTok{  }\KeywordTok{facet_wrap}\NormalTok{(}\OperatorTok{~}\NormalTok{bigregion)}\OperatorTok{+}
\StringTok{  }\KeywordTok{theme_bw}\NormalTok{()}\OperatorTok{+}
\StringTok{  }\NormalTok{theme}\OperatorTok{+}
\StringTok{  }\KeywordTok{labs}\NormalTok{(}\DataTypeTok{x=}\StringTok{"Political Views"}\NormalTok{, }\DataTypeTok{y=}\StringTok{"Frequency"}\NormalTok{, }
       \DataTypeTok{title=}\StringTok{"Political views of the respondents!"}\NormalTok{, }\DataTypeTok{subtitle=}\StringTok{"Bigregion wise"}\NormalTok{, }
       \DataTypeTok{caption=}\StringTok{"Dataset:gss_sm"}\NormalTok{)}\OperatorTok{+}
\StringTok{  }\KeywordTok{theme}\NormalTok{(}\DataTypeTok{axis.text.x =} \KeywordTok{element_text}\NormalTok{(}\DataTypeTok{angle =} \DecValTok{45}\NormalTok{, }\DataTypeTok{hjust =} \DecValTok{1}\NormalTok{))}
\end{Highlighting}
\end{Shaded}

\includegraphics{194161020_files/figure-latex/unnamed-chunk-5-2.pdf}

\hypertarget{question-5}{%
\section{Question 5}\label{question-5}}

Plot following from the gss sm data.

\textbf{I. Histogram. Showcasing ages into bins.}

\begin{Shaded}
\begin{Highlighting}[]
\KeywordTok{ggplot}\NormalTok{(}\DataTypeTok{data =}\NormalTok{ gss_sm) }\OperatorTok{+}\StringTok{ }
\StringTok{  }\KeywordTok{geom_histogram}\NormalTok{(}\DataTypeTok{mapping =} \KeywordTok{aes}\NormalTok{(}\DataTypeTok{x =}\NormalTok{ age),}
                 \DataTypeTok{binwidth =} \DecValTok{1}\NormalTok{, }\DataTypeTok{fill=}\StringTok{"green"}\NormalTok{, }\DataTypeTok{alpha=}\FloatTok{0.35}\NormalTok{)}\OperatorTok{+}
\StringTok{  }\KeywordTok{labs}\NormalTok{(}\DataTypeTok{x=}\StringTok{"Age"}\NormalTok{, }\DataTypeTok{y=}\StringTok{"Count"}\NormalTok{, }
       \DataTypeTok{title=}\StringTok{"Age of the Respondents!"}\NormalTok{, }
       \DataTypeTok{caption=}\StringTok{"Dataset:gss_sm"}\NormalTok{)}\OperatorTok{+}\KeywordTok{theme_classic}\NormalTok{()}\OperatorTok{+}\NormalTok{theme}
\end{Highlighting}
\end{Shaded}

\includegraphics{194161020_files/figure-latex/unnamed-chunk-6-1.pdf} **
II. Modify the previous histogram, dividing the observation based on
variable race. Each race should be represented in a different color.**

\begin{Shaded}
\begin{Highlighting}[]
\KeywordTok{ggplot}\NormalTok{(}\DataTypeTok{data =}\NormalTok{ gss_sm) }\OperatorTok{+}\StringTok{ }
\StringTok{  }\KeywordTok{geom_histogram}\NormalTok{(}\DataTypeTok{mapping =} \KeywordTok{aes}\NormalTok{(}\DataTypeTok{x =}\NormalTok{ age, }\DataTypeTok{fill=}\NormalTok{race), }
                 \DataTypeTok{binwidth =} \DecValTok{1}\NormalTok{, }\DataTypeTok{alpha=}\FloatTok{0.6}\NormalTok{)}\OperatorTok{+}
\StringTok{  }\KeywordTok{labs}\NormalTok{(}\DataTypeTok{x=}\StringTok{"Age"}\NormalTok{, }\DataTypeTok{y=}\StringTok{"Count"}\NormalTok{, }\DataTypeTok{title=}\StringTok{"Age of the Respondents!"}\NormalTok{, }
       \DataTypeTok{subtitle=}\StringTok{"Race wise"}\NormalTok{, }\DataTypeTok{caption=}\StringTok{"Dataset:gss_sm"}\NormalTok{)}\OperatorTok{+}\KeywordTok{theme_classic}\NormalTok{()}\OperatorTok{+}\NormalTok{theme}
\end{Highlighting}
\end{Shaded}

\includegraphics{194161020_files/figure-latex/unnamed-chunk-7-1.pdf}
\textbf{III. 2 Stacked Density plots, dividing the observation based on
the variable race:} \textbf{-- In x-axis, choose variable age}
\textbf{-- In x-axis, choose variable agegrp.} \textbf{Discuss which
variable among the two age and agegrp, is more appropriate for plots
like a histogram and density plots and why?}

\begin{Shaded}
\begin{Highlighting}[]
\NormalTok{p <-}\StringTok{ }\KeywordTok{ggplot}\NormalTok{(}\DataTypeTok{data =}\NormalTok{ gss_sm,}
            \DataTypeTok{mapping =} \KeywordTok{aes}\NormalTok{(}\DataTypeTok{x =}\NormalTok{ age, }\DataTypeTok{fill=}\NormalTok{race))}
\NormalTok{p }\OperatorTok{+}\StringTok{ }\KeywordTok{geom_density}\NormalTok{(}\DataTypeTok{alpha =} \FloatTok{0.3}\NormalTok{)}\OperatorTok{+}
\StringTok{  }\KeywordTok{theme_classic}\NormalTok{()}\OperatorTok{+}
\StringTok{  }\NormalTok{theme}\OperatorTok{+}
\StringTok{  }\KeywordTok{labs}\NormalTok{(}\DataTypeTok{x=}\StringTok{"Age"}\NormalTok{, }\DataTypeTok{y=}\StringTok{"Density"}\NormalTok{, }
       \DataTypeTok{title=}\StringTok{"2 Stack Density plot for Age"}\NormalTok{,}
       \DataTypeTok{caption=}\StringTok{"Dataset:gss_sm"}\NormalTok{)}
\end{Highlighting}
\end{Shaded}

\includegraphics{194161020_files/figure-latex/unnamed-chunk-8-1.pdf}

\begin{Shaded}
\begin{Highlighting}[]
\NormalTok{p <-}\StringTok{ }\KeywordTok{ggplot}\NormalTok{(}\DataTypeTok{data =}\NormalTok{ gss_sm,}
            \DataTypeTok{mapping =} \KeywordTok{aes}\NormalTok{(}\DataTypeTok{x =}\NormalTok{ agegrp, }\DataTypeTok{fill=}\NormalTok{race))}
\NormalTok{p }\OperatorTok{+}\StringTok{ }\KeywordTok{geom_density}\NormalTok{(}\DataTypeTok{alpha =} \FloatTok{0.3}\NormalTok{)}\OperatorTok{+}
\StringTok{  }\KeywordTok{theme_classic}\NormalTok{()}\OperatorTok{+}
\StringTok{  }\NormalTok{theme}\OperatorTok{+}
\StringTok{  }\KeywordTok{labs}\NormalTok{(}\DataTypeTok{x=}\StringTok{"Agegrp"}\NormalTok{, }\DataTypeTok{y=}\StringTok{"Density"}\NormalTok{, }
       \DataTypeTok{title=}\StringTok{"2 Stack Density plot for Agegrp"}\NormalTok{, }
       \DataTypeTok{subtitle =} \StringTok{"Race wise"}\NormalTok{, }
       \DataTypeTok{caption=}\StringTok{"Dataset:gss_sm"}\NormalTok{)}
\end{Highlighting}
\end{Shaded}

\includegraphics{194161020_files/figure-latex/unnamed-chunk-8-2.pdf}

\begin{Shaded}
\begin{Highlighting}[]
\NormalTok{p <-}\StringTok{ }\KeywordTok{ggplot}\NormalTok{(}\DataTypeTok{data =}\NormalTok{ gss_sm,}
            \DataTypeTok{mapping =} \KeywordTok{aes}\NormalTok{(}\DataTypeTok{x =}\NormalTok{ agegrp, }\DataTypeTok{fill=}\NormalTok{race))}
\NormalTok{p }\OperatorTok{+}\StringTok{ }\KeywordTok{geom_histogram}\NormalTok{(}\DataTypeTok{alpha =} \FloatTok{0.5}\NormalTok{,}\DataTypeTok{stat=}\StringTok{"count"}\NormalTok{)}\OperatorTok{+}
\StringTok{  }\KeywordTok{theme_classic}\NormalTok{()}\OperatorTok{+}
\StringTok{  }\NormalTok{theme}\OperatorTok{+}
\StringTok{  }\KeywordTok{labs}\NormalTok{(}\DataTypeTok{x=}\StringTok{"Agegrp"}\NormalTok{, }\DataTypeTok{y=}\StringTok{"Count"}\NormalTok{, }
       \DataTypeTok{title=}\StringTok{"Histogram for Agegrp"}\NormalTok{, }
       \DataTypeTok{subtitle =} \StringTok{"Race wise"}\NormalTok{, }
       \DataTypeTok{caption=}\StringTok{"Dataset:gss_sm"}\NormalTok{)}
\end{Highlighting}
\end{Shaded}

\includegraphics{194161020_files/figure-latex/unnamed-chunk-8-3.pdf}
Agegrp is a discrete variable. It serves as a suitable variable for the
histogram, while when plotted as a density plot , it doesnot give
discriminative observation.

\textbf{IV. Density plot} \textbf{-- x-axis: age.} \textbf{-- Division
based on variable degree.} \textbf{-- Faceting based on variable
income16.}

\begin{Shaded}
\begin{Highlighting}[]
\KeywordTok{ggplot}\NormalTok{(}\DataTypeTok{data =}\NormalTok{ gss_sm,}\DataTypeTok{mapping =} \KeywordTok{aes}\NormalTok{(}\DataTypeTok{x =}\NormalTok{ age )) }\OperatorTok{+}\StringTok{ }
\StringTok{  }\KeywordTok{geom_density}\NormalTok{(}\KeywordTok{aes}\NormalTok{(}\DataTypeTok{fill=}\NormalTok{degree),}\DataTypeTok{alpha=}\FloatTok{0.5}\NormalTok{)}\OperatorTok{+}
\StringTok{  }\KeywordTok{facet_wrap}\NormalTok{(}\OperatorTok{~}\NormalTok{income16)}\OperatorTok{+}
\StringTok{  }\KeywordTok{theme_bw}\NormalTok{()}\OperatorTok{+}
\StringTok{  }\NormalTok{theme}\OperatorTok{+}
\StringTok{  }\KeywordTok{labs}\NormalTok{(}\DataTypeTok{x=}\StringTok{"Age"}\NormalTok{, }\DataTypeTok{y=}\StringTok{"Density"}\NormalTok{,}
       \DataTypeTok{title=}\StringTok{"Density plot of the Respondents's age!"}\NormalTok{, }
       \DataTypeTok{caption=}\StringTok{"Dataset: gss_sm"}\NormalTok{, }
       \DataTypeTok{subtitle=}\StringTok{"Income wise"}\NormalTok{)}
\end{Highlighting}
\end{Shaded}

\includegraphics{194161020_files/figure-latex/unnamed-chunk-9-1.pdf}

\begin{center}\rule{0.5\linewidth}{0.5pt}\end{center}

\end{document}
